% TODO: Длинные заголовки при переносе строки продолжаются под цифрами. Т.е. не начало текста не выровнено
% ----------------------------- БАЗОВЫЕ НАСТРОЙКИ  -----------------------------

\documentclass[oneside, final, 14pt,  a4paper]{extreport}    % Класс документа. Чаще всего использует его
\usepackage[T2A]{fontenc}
\usepackage[utf8]{inputenc}                        % Кодировка utf8x для linux
\usepackage[english,russian]{babel}                % Переносы и прочее для русского и английского
\usepackage[linktocpage=true]{hyperref}            % Гиперссылки
\usepackage{indentfirst}                           % Красная строка для первых абзацев
\usepackage{mathtools}
\usepackage{tabularx}                              % Много табличной боли и магии
%\usepackage{ltablex}                              % Длинные таблицы с разрывом страниц
%\keepXColumns                                     % Чтобы ltablex не ломал ширину
\usepackage{tabulary}                              % Для создания выравненных "где..." для формул
\usepackage{graphicx}
\usepackage{titlesec}                              % Для настройки названий глав, разделов итп
\usepackage{titletoc}                              % Для настройки оглавления
\usepackage{enumitem}                              % Для настройки списков
\usepackage{flafter}                               % Плавающие элементы встречаются после ссылки на них
\usepackage{caption}                               % Настройка подписей плавающих элементов
\usepackage{setspace}                              % Для настройки интервалов
\usepackage{color, soulutf8}                       % Выделенный текст
\usepackage{csquotes}
\usepackage{algorithm}                             % Псевдокод
\usepackage{algpseudocode}
\usepackage{chngcntr}                              % Чтоб настроить сквозную нумерацию
\usepackage{lastpage}                              % Получить количество страниц
\usepackage[figure,table]{totalcount}              % Получить количество рисунков и таблиц

% Библиография
\usepackage[
    backend=biber,
    sorting=none,                                  % Сортировка в порядке цитирования
	style=gost-numeric,
	language=auto,                                 % Автовыбор стиля, напр. писать [et al] вместо [и др]
	autolang=other                                 % для англоязычных публикаций (langid={english})
] {biblatex}

\linespread{1.3}                                   % Полуторный интервал

% Настройка полей
\usepackage{geometry}
\geometry{left=3cm}
\geometry{right=1cm}
\geometry{top=1.5cm}
\geometry{bottom=2cm}


\sloppy                                           % Избегать залезания строк на поля (надо?)
\setlength\parindent{1.5cm}                       % Отступ красной строки

\newcommand{\nf}{\normalfont}

% Для tabularx: как X (растянутый), только выравненный по центру
\newcolumntype{Y}{>{\centering\arraybackslash}X}
\newcolumntype{P}{>{\raggedleft\arraybackslash}X}

%--------------------------- НАСТРОЙКА ТИТУЛЬНОГО ЛИСТА ------------------------
\newenvironment{nospasing}
{
    \begin{spacing}{1}
}
{
    \end{spacing}
}

% Поля титульного листа с подписями и линями (подпись, ФИО итп)

% подпись под пустой линией заданной длины
\newcommand{\efield}[2][2cm]{
	%1 - ширина поля
	%2 - подпись под линией
	$\underset{\text{(#2)}}{\underline{\hspace*{#1}}}$
}

% Подпись под текстом
\newcommand{\tfield}[2]{
	%1 - содержимое поля
	%2 - подпись под линией
	$\underset{\text{(#2)}}{\underline{\smash{\text{#1}}}}$
}

% Название поля, линия до конца строки, значение поля над линией, подпись под линией

\newcommand{\lfield}[3]{
	%1 - название поля
	%2 - содержимое поля
	%3 - подпись под линией
	\noindent
	\renewcommand{\arraystretch}{0.7}
	\begin{tabularx}{\linewidth}{@{}lY@{}}
    	#1 & #2 \\
    	\cline{2-2}
           & \footnotesize(#3)\normalsize
	\end{tabularx}

}


% ----------------------------- НАСТРОЙКИ ЗАГЛАВИЙ -----------------------------
%% Отступ 1.5 слева (как у красной строки)t
% Нет точки между номером и названием
% Интервал между подзаголовками 1.5
% Интервал между заголовком и текстом 2*1.5
% Поддержка приложений

% Глава
\titleformat{\chapter}
	[block]                % Shape. block убирает перенос заглвания на новую строку
    {\normalfont}          % Format. Собственно, стиль
    {\thechapter}          % Label. Номер главы.
    {8pt}                  % Sep. Пробел между номером и главой (TODO: уточнить)
    {}                     % before-code. Код перед названием
\titlespacing*{\chapter}
	{1.5cm}                % Левый отступ (как у красной строки)
	{18pt}                 % Верхний отступ, 1.5 интервал
	{18pt}                 % Нижний отступ, 1.5 интервал

% Раздел
\titleformat{\section}
	{\normalfont}
	{\thesection}
	{8pt}{}
\titlespacing*{\section}
	{1.5cm}{18pt}{18pt}

% Подраздел
\titleformat{\subsection}
	{\normalfont}
	{\thesubsection}
	{8pt}{}
\titlespacing*{\subsection}
	{\parindent}{18pt}{18pt}

% Глава без номера (введение, заключение и т.п.)
\newcommand{\nnchapter}[1]
{
	\chapter*{#1}
	\addcontentsline{toc}{chapter}{#1}
}

% Фейковая глава для автореферата
% Зачем платить больше, если не нужно содержание?
\newcommand{\referchapter}[1]
{

    \vspace{18pt}
    #1
    \vspace{18pt}

}


% Приложения
% Использовать \chapter{} для создания приложений
% Очень грязный хак, но работает
\newcommand{\StartAppendix}
{
	\setcounter{chapter}{0}
}

\renewcommand{\appendix}[1]
{
	\newpage
	\stepcounter{chapter}
	\newcommand{\theappendix}{ПРИЛОЖЕНИЕ \MakeUppercase{\asbuk{chapter}}}
	\addcontentsline{toc}{chapter}{\texorpdfstring{\theappendix} ~--- #1}
	\begin{center}
		\theappendix\\
		{#1}
	\end{center}
}

% Расстояние между заглавиями и текстом должно быть 2 полуторных интервала,
% а расстояние между заглавиями - один полуторный интервал.
% Не придумал ничего лучше, кроме как вставлять вручную
\newcommand{\aftertitle}{\vskip 18pt}

% ----------------------------- НАСТРОЙКИ СОДЕРЖАНИЯ ---------------------------
% Нет выделения жирным
% Все с одним уровнем отступа
% Поддержка приложений

% Главы
\titlecontents{chapter}
	[0em] {}
	{\thecontentslabel~}{}
	{\titlerule*[1pc]{.}\contentspage}

% Разделы
\titlecontents{section}
	[0em] {}
	{\thecontentslabel~}{}
	{\titlerule*[1pc]{.}\contentspage}

% Подразделы
\titlecontents{subsection}
	[0em] {}
	{\thecontentslabel~}{}
	{\titlerule*[1pc]{.}\contentspage}

% Заголовок
\addto\captionsrussian{
	\renewcommand{\contentsname} {СОДЕРЖАНИЕ}
}

%-------------------------------- НАСТРОЙКИ СПИСКОВ ----------------------------
% Маркерный список
\setlist[itemize]{
	label=-,                  % Дефис в каяестве маркера
	leftmargin=1.5cm,         % Текст в списке выравнен по красной строке
	itemindent=15pt,          % Маркер выравнен по красной строке, т.е. первая строка чуть сдвинута на размер маркера
	nosep                     % Убираем интервал между пунктами списков
}

% Числовой
\setlist[enumerate]{
    label*=\arabic*),
    leftmargin=1.5cm,
    itemindent=15pt,
    nosep
}

%--------------------------- НАСТРОЙКИ РИСУНКОВ И ТАБЛИЦ -----------------------
% Рисунки подписываются "Рисунок N - ..." по центру
% Таблицы подписываются "Таблица N - ..." с левого края

\captionsetup[figure]{name=Рисунок, labelsep=endash, justification=centering}
\captionsetup[table]{name=Таблица, labelsep=endash, justification=raggedright, singlelinecheck=false}

% Сквозная нумерация таблиц, рисунков и формул
\counterwithout{figure}{chapter}
\counterwithout{table}{chapter}
\counterwithout{equation}{chapter}
\pdfimageresolution=150

%---------------------------------- ФОРМУЛЫ ------------------------------------

\newcommand{\degsym}{^{\circ}}    % Градус
\newcommand{\CST}{\mathcal{C}}    % C-State, пространство конфигурации
\newcommand{\XST}{\mathcal{X}}    % X-State, пространство состояний
\newcommand{\vect}[1]{\overrightarrow{#1}}
\DeclarePairedDelimiter\floor{\lfloor}{\rfloor}

%-------------------------------- БИБЛИОГРАФИЯ ---------------------------------

\addbibresource{autonomouscar.bib}
\begin{document}

\begin{center}
	АННОТАЦИЯ
\end{center}

\begin{center}
	ABSTRACT
\end{center}

\tableofcontents
\nnchapter{ВВЕДЕНИЕ (ГЛАВА БЕЗ НУМЕРАЦИИ)}
\aftertitle

Это демонстрация настройки стилей, соответствующих требования магистерской диссертации ВолгГТУ. 

Это еще один очередной стиль для оформления магистреских диссертаций в \LaTeX, который лишен ряда \bf фатальных
недостатков \nf других стилей и пакетов. Полностью соответствует требованиям ВолгГТУ вообще и кафедры ЭВМиС в
частности, насколько это вообще возможно, по причине того, что у нормоконтроллера семь пятниц на неделе. 

Преимущества:
\begin{itemize}
	\item соответствует требованиям,
	\item не использует монструозные пакеты, вроде diser, в которых слишком много всего,
	      разобраться и настроить которые под точные требования может быть сложно,
	\item нет никакой внутренней магии, наверное, 
	\item все настройки заданы явным образом, с подробными комментариями, так что этот
	      стиль легко исправить, когда нормоконтроллер опять к чему-то придерется.
\end{itemize}


\chapter{ОСНОВНЫЕ ВОЗМОЖНОСТИ (ГЛАВА С НУМЕРАЦИЕЙ)}
\section{Базовая верстка}
\aftertitle

Стиль основан на классе extreport, но исправляет ряд его отличий от требований. Точнее, явным образом
переопределяются все (почти все) настройки, мало ли что. Например, по-умолчанию разделы верхнего уровня
(которые chapter), описывались как "Глава 1" большим кеглем, а на новой строке название. Базовая верстка
соответствует требования на формат страницы, шрифт, поля, разделы и т.п.:
\begin{itemize}
	\item шрифт похож на Times New Roman, 14 кегль;
	\item полуторный интервал;
	\item поля: левое 30 мм, право 10 мм, верхнее 15, нижнее 20;
	\item заглавия разделов с абзацным отступом, точка после номера раздела не ставится;
	\item поддержка разделов без заглавия (например, введение, заключение);
	\item расстояние между заголовками раздела и подраздела ~--- один полуторный интервал, между 
	      разделом и текстом ~--- два полуторных интервала (с помощью ручного костыля).
	\item правильное содержание
\end{itemize}

\section{Списки}
\aftertitle

Возможности списков:
\begin{itemize}
	\item дефис в качестве маркера;
	\item текст выравнен по красной строке;
	\item маркер также выравнен по красной строке, таким образом в случае длинного текста, который переносится
	      на несколько строк, первая строка оказывается немного сдвинутой на размер маркера. Это ужасное требование,
	      но щито поделать;
	\item легко настроить и более ужасное требование - чтобы в случае переноса длинного текста, он выравнивался
	      совсем по левому краю;
	\item уменьшенный интервал между пунктами списка.
\end{itemize}

\section{Рисунки}
\aftertitle

Рисунки имеют сквозную нумерацию и правильный формат подписи. Это пример рисунка. На рисунке \ref{img:shepard}
 изображен мем с  капитаном Шепардом. Рисунки вставляются обязательно после ссылки на них, но конкретное положение
 зависит от милости богов латеха. Для того, чтобы указать более конкретное расположение, можно использовать параметр [h].
\begin{figure}[h]
	\centering
	\includegraphics{images/shepard}
	\caption{Капитан Шепард}
	\label{img:shepard}
\end{figure}

Также можно делать несколько рисунков в одном, как это представлено на рисунке \ref{img:multiple}.

\begin{figure}[h]
	\begin{minipage}[h]{0.5\linewidth}
		\center{\includegraphics{images/shepard} \\ а)}
	\end{minipage}
	\begin{minipage}[h]{0.5\linewidth}
		\center{\includegraphics{images/shepard} \\ b)}
	\end{minipage}
	\caption{Несколько рисунков в одном}
	\label{img:multiple}
\end{figure}

\section{Таблицы}
\aftertitle

Таблицы, как и рисунки, - плавающие элементы, поэтому с ними все точно так же, как и с рисунками. Алсо, существует
большое количество типов таблиц, tabular, tabular*, tabularx, tabulary, longtable... Мы отобрали лучшие варианты, а
остальные прислали вам.

В таблице \ref{tab:memes} приведена реакция испытуемых на различные мемасы.
\begin{table}[h!]
	\caption{простая таблица}
	\label{tab:memes}
	\begin{tabularx}{\textwidth}{|l|X|X|X|X|}
		\hline
		        & Сложный прекол & Ор & Лютый гнаарский ор \\\hline
		Мем А   & 30\%           & 146\%      & 5\%        \\\hline
		Мем Б   & 20\%           & 14.88\%    & овер9000\% \\\hline
		Мем В   &  0\%           & 0\%        & 0\%        \\\hline
		Мем Г   &  0\%           & 0\%        & 0\%        \\\hline				
	\end{tabularx}
\end{table}

А таблица \ref{tab:long} такая длинная, что переносится на другую страницу. 

%\begin{table}[h!]
%	\caption{длинная таблица}
	\begin{tabularx}{\textwidth}{|X|X|X|}
		\caption{длинная таблица}\label{tab:long} \\
		\hline
		Row A & Row B & Row C\\
		\hline
		\hline
		\endfirsthead
			
		\multicolumn{3}{l}{Продолжение таблицы \thetable}\\
		\hline
		\endhead
		
		\endfoot
		
		\hline
		\endlastfoot
		
		A & B & C\\
		A & B & C\\
		A & B & C\\
		A & B & C\\
		A & B & C\\
		A & B & C\\
		A & B & C\\
		A & B & C\\
		A & B & C\\
		A & B & C\\
		A & B & C\\
		A & B & C\\
	\end{tabularx}
%\end{table}

\section{Формулы}
\aftertitle
Пример вынесенной формулы:
\begin{equation}
\label{eq:cost_function}
J = \int_{0}^{T}{\dddot{x}dT}
\end{equation}
\begin{tabulary}{\linewidth}{LLLL}
	где & $T$ &~---& время маневра, которое очень очень очень очень очень очень очень очень  длинный текст, \\
	& $x$ &~---& координата автомобиля.
\end{tabulary}

Пример ссылки на формулу:  (\ref{eq:cost_function}).

\section{Подраздел с очень очень очень очень очень очень очень очень очень очень очень очень очень очень очень длинным названием}

\StartAppendix

\appendix{ИСХОДНЫЙ КОД ПРОГРАММЫ}

\end{document}