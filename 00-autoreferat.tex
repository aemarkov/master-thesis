% TODO: Длинные заголовки при переносе строки продолжаются под цифрами. Т.е. не начало текста не выровнено
% ----------------------------- БАЗОВЫЕ НАСТРОЙКИ  -----------------------------

\documentclass[oneside, final, 14pt,  a4paper]{extreport}    % Класс документа. Чаще всего использует его
\usepackage[T2A]{fontenc}
\usepackage[utf8]{inputenc}                        % Кодировка utf8x для linux
\usepackage[english,russian]{babel}                % Переносы и прочее для русского и английского
\usepackage[linktocpage=true]{hyperref}            % Гиперссылки
\usepackage{indentfirst}                           % Красная строка для первых абзацев
\usepackage{mathtools}
\usepackage{tabularx}                              % Много табличной боли и магии
%\usepackage{ltablex}                              % Длинные таблицы с разрывом страниц
%\keepXColumns                                     % Чтобы ltablex не ломал ширину
\usepackage{tabulary}                              % Для создания выравненных "где..." для формул
\usepackage{graphicx}
\usepackage{titlesec}                              % Для настройки названий глав, разделов итп
\usepackage{titletoc}                              % Для настройки оглавления
\usepackage{enumitem}                              % Для настройки списков
\usepackage{flafter}                               % Плавающие элементы встречаются после ссылки на них
\usepackage{caption}                               % Настройка подписей плавающих элементов
\usepackage{setspace}                              % Для настройки интервалов
\usepackage{color, soulutf8}                       % Выделенный текст
\usepackage{csquotes}
\usepackage{algorithm}                             % Псевдокод
\usepackage{algpseudocode}
\usepackage{chngcntr}                              % Чтоб настроить сквозную нумерацию
\usepackage{lastpage}                              % Получить количество страниц
\usepackage[figure,table]{totalcount}              % Получить количество рисунков и таблиц

% Библиография
\usepackage[
    backend=biber,
    sorting=none,                                  % Сортировка в порядке цитирования
	style=gost-numeric,
	language=auto,                                 % Автовыбор стиля, напр. писать [et al] вместо [и др]
	autolang=other                                 % для англоязычных публикаций (langid={english})
] {biblatex}

\linespread{1.3}                                   % Полуторный интервал

% Настройка полей
\usepackage{geometry}
\geometry{left=3cm}
\geometry{right=1cm}
\geometry{top=1.5cm}
\geometry{bottom=2cm}


\sloppy                                           % Избегать залезания строк на поля (надо?)
\setlength\parindent{1.5cm}                       % Отступ красной строки

\newcommand{\nf}{\normalfont}

% Для tabularx: как X (растянутый), только выравненный по центру
\newcolumntype{Y}{>{\centering\arraybackslash}X}
\newcolumntype{P}{>{\raggedleft\arraybackslash}X}

%--------------------------- НАСТРОЙКА ТИТУЛЬНОГО ЛИСТА ------------------------
\newenvironment{nospasing}
{
    \begin{spacing}{1}
}
{
    \end{spacing}
}

% Поля титульного листа с подписями и линями (подпись, ФИО итп)

% подпись под пустой линией заданной длины
\newcommand{\efield}[2][2cm]{
	%1 - ширина поля
	%2 - подпись под линией
	$\underset{\text{(#2)}}{\underline{\hspace*{#1}}}$
}

% Подпись под текстом
\newcommand{\tfield}[2]{
	%1 - содержимое поля
	%2 - подпись под линией
	$\underset{\text{(#2)}}{\underline{\smash{\text{#1}}}}$
}

% Название поля, линия до конца строки, значение поля над линией, подпись под линией

\newcommand{\lfield}[3]{
	%1 - название поля
	%2 - содержимое поля
	%3 - подпись под линией
	\noindent
	\renewcommand{\arraystretch}{0.7}
	\begin{tabularx}{\linewidth}{@{}lY@{}}
    	#1 & #2 \\
    	\cline{2-2}
           & \footnotesize(#3)\normalsize
	\end{tabularx}

}


% ----------------------------- НАСТРОЙКИ ЗАГЛАВИЙ -----------------------------
%% Отступ 1.5 слева (как у красной строки)t
% Нет точки между номером и названием
% Интервал между подзаголовками 1.5
% Интервал между заголовком и текстом 2*1.5
% Поддержка приложений

% Глава
\titleformat{\chapter}
	[block]                % Shape. block убирает перенос заглвания на новую строку
    {\normalfont}          % Format. Собственно, стиль
    {\thechapter}          % Label. Номер главы.
    {8pt}                  % Sep. Пробел между номером и главой (TODO: уточнить)
    {}                     % before-code. Код перед названием
\titlespacing*{\chapter}
	{1.5cm}                % Левый отступ (как у красной строки)
	{18pt}                 % Верхний отступ, 1.5 интервал
	{18pt}                 % Нижний отступ, 1.5 интервал

% Раздел
\titleformat{\section}
	{\normalfont}
	{\thesection}
	{8pt}{}
\titlespacing*{\section}
	{1.5cm}{18pt}{18pt}

% Подраздел
\titleformat{\subsection}
	{\normalfont}
	{\thesubsection}
	{8pt}{}
\titlespacing*{\subsection}
	{\parindent}{18pt}{18pt}

% Глава без номера (введение, заключение и т.п.)
\newcommand{\nnchapter}[1]
{
	\chapter*{#1}
	\addcontentsline{toc}{chapter}{#1}
}

% Фейковая глава для автореферата
% Зачем платить больше, если не нужно содержание?
\newcommand{\referchapter}[1]
{

    \vspace{18pt}
    #1
    \vspace{18pt}

}


% Приложения
% Использовать \chapter{} для создания приложений
% Очень грязный хак, но работает
\newcommand{\StartAppendix}
{
	\setcounter{chapter}{0}
}

\renewcommand{\appendix}[1]
{
	\newpage
	\stepcounter{chapter}
	\newcommand{\theappendix}{ПРИЛОЖЕНИЕ \MakeUppercase{\asbuk{chapter}}}
	\addcontentsline{toc}{chapter}{\texorpdfstring{\theappendix} ~--- #1}
	\begin{center}
		\theappendix\\
		{#1}
	\end{center}
}

% Расстояние между заглавиями и текстом должно быть 2 полуторных интервала,
% а расстояние между заглавиями - один полуторный интервал.
% Не придумал ничего лучше, кроме как вставлять вручную
\newcommand{\aftertitle}{\vskip 18pt}

% ----------------------------- НАСТРОЙКИ СОДЕРЖАНИЯ ---------------------------
% Нет выделения жирным
% Все с одним уровнем отступа
% Поддержка приложений

% Главы
\titlecontents{chapter}
	[0em] {}
	{\thecontentslabel~}{}
	{\titlerule*[1pc]{.}\contentspage}

% Разделы
\titlecontents{section}
	[0em] {}
	{\thecontentslabel~}{}
	{\titlerule*[1pc]{.}\contentspage}

% Подразделы
\titlecontents{subsection}
	[0em] {}
	{\thecontentslabel~}{}
	{\titlerule*[1pc]{.}\contentspage}

% Заголовок
\addto\captionsrussian{
	\renewcommand{\contentsname} {СОДЕРЖАНИЕ}
}

%-------------------------------- НАСТРОЙКИ СПИСКОВ ----------------------------
% Маркерный список
\setlist[itemize]{
	label=-,                  % Дефис в каяестве маркера
	leftmargin=1.5cm,         % Текст в списке выравнен по красной строке
	itemindent=15pt,          % Маркер выравнен по красной строке, т.е. первая строка чуть сдвинута на размер маркера
	nosep                     % Убираем интервал между пунктами списков
}

% Числовой
\setlist[enumerate]{
    label*=\arabic*),
    leftmargin=1.5cm,
    itemindent=15pt,
    nosep
}

%--------------------------- НАСТРОЙКИ РИСУНКОВ И ТАБЛИЦ -----------------------
% Рисунки подписываются "Рисунок N - ..." по центру
% Таблицы подписываются "Таблица N - ..." с левого края

\captionsetup[figure]{name=Рисунок, labelsep=endash, justification=centering}
\captionsetup[table]{name=Таблица, labelsep=endash, justification=raggedright, singlelinecheck=false}

% Сквозная нумерация таблиц, рисунков и формул
\counterwithout{figure}{chapter}
\counterwithout{table}{chapter}
\counterwithout{equation}{chapter}
\pdfimageresolution=150

%---------------------------------- ФОРМУЛЫ ------------------------------------

\newcommand{\degsym}{^{\circ}}    % Градус
\newcommand{\CST}{\mathcal{C}}    % C-State, пространство конфигурации
\newcommand{\XST}{\mathcal{X}}    % X-State, пространство состояний
\newcommand{\vect}[1]{\overrightarrow{#1}}
\DeclarePairedDelimiter\floor{\lfloor}{\rfloor}

%-------------------------------- БИБЛИОГРАФИЯ ---------------------------------

\addbibresource{autonomouscar.bib}
\begin{document}

\begin{nospasing}
\begin{center}
    Министерство образования и науки Российской Федерации \\
    Федеральное государственное бюджетное образовательное учреждение \\
    высшего образования \\
    «Волгоградский государственный технический университет» \\
    Факультет электроники и вычислительной техники \\
    Кафедра «Электронно-вычислительные машины и системы» \\

    \begin{flushright}
    На правах рукописи \\
    \end{flushright}
    
    МАРКОВ АЛЕКСЕЙ ЕВГЕНЬЕВИЧ \\
    \vspace{1cm}
    \hl{???}\\
    \vspace{1cm}
    Магистерская программа \\
    «Высокопроизводительные вычислительные системы» \\
    Направление подготовки 09.04.01 «Информатика и вычислительная техника» \\
    \vspace{1cm}
    АВТОРЕФЕРАТ \\
    \vspace{1cm}
    диссертации на соискание степени \\
    магистра \\
    
    \vfill
    
    Волгоград – 2019
\end{center}
\end{nospasing}

\newpage

\begin{nospasing}
\noindent
Работа выполнена на кафедре «Электронно-вычислительные машины и
системы» Волгоградского государственного технического университета\\

\noindent
Научный руководитель:\\

    \setlength{\leftskip}{3cm}
    \noindent
    доктор технических наук, \\
    профессор Горобцов Александр Сергеевич \\	
    
    \setlength{\leftskip}{0pt}

\noindent
Рецензент:

    \setlength{\leftskip}{3cm}
    \noindent
    \hl{TODO} \\	
    
    \setlength{\leftskip}{0pt}

\end{nospasing}

\newpage

%%%%%%%%%%%%%%%%%%%%%%%%%%%%%%%%%%%%%%%%%%%%%%%%%%%%%%%%%%%%%%%%%%%%%%%%%%%%%%

\referchapter{ОБЩАЯ ХАРАКТЕРИСТИКА РАБОТЫ}
\textbf{Актуальность темы исследования.}

Использование беспилотных систем и интеллектуальных систем помощи водителю (ADAS) на дорогах общего
пользования приведет к повышению безопасности дорожного движения по причине уменьшения человеческого
фактора, увеличению пропускной способности дорог и дорожной инфраструктуры, уменьшению пробок, а также
уменьшить количество личных автомобилей при сохранении текущих сценариев использования, что еще
сильнее снизит нагрузку на инфраструктуру и загрязнение окружающей среды. 

На кафедрах ЭВМиС, ВМ и АТ ВолгГТУ в настоящее время ведется разработка прототипа беспилотного
автомобиля на базе автомобиля Лада Калина. 

Основным результатом проводимых исследований является \hl{TODO}

\textbf{Цель и задачи работы.}

Целью данной работы является разработка системы управления движением беспилотного автомобиля.
Для достижения данной цели решались следующие задачи:
\begin{enumerate}
    \item анализ существующих подходов к задаче управления беспилотными автомобилями в целом, выделение 
    типичных подсистем,
    \item проектирование подсистем управления движением беспилотным автомобилем,
    \item реализация подсистем управления движением,
    \item проведение экспериментов и оценка результатов работы. 
\end{enumerate}

\textbf{Объектом исследования является} процесс управления движением беспилотного автомобиля.

\textbf{Предметом исследования являются} алгоритмы и подходы к реализации системы управления движением
беспилотного автомобиля.

\textbf{Основные положения, выносимые на защиту:}
\begin{enumerate}
    \item подход к локальному планированию траектории движения беспилотного автомобиля с помощью
          полиномов пятого порядка в системе координат Френе \hl{TODO: уточнить перевод},
    \item разработанная архитектура системы управления беспилотным автомобилем.
\end{enumerate}

\textbf{Научной новизной} работы является \hl{TODO}

\textbf{Практическая ценность} работы заключается в:
\begin{enumerate}
    \item программной реализации системы управления движением беспилотного автомобиля,
    \item \hl{...}
    \item \hl{...}
\end{enumerate}

\textbf{Апробация} работы проводилась \hl{TODO}

\textbf{Публикации.} По материалам диссертации автором \hl{TODO: ничего не было опубликовано}

\textbf{Структура и объем работы}. Диссертационная работа состоит из введения, четырех глав, заключения, 
списка использованных источников из \hl{23} наименований и насчитывает \hl{TODO} страниц.

\referchapter{ОСНОВНОЕ СОДЕРЖАНИЕ РАБОТЫ}
\textbf{Во введении} обосновывается выбор темы диссертационного исследования и ее актуальность,
определяются цели и задачи работы, объект и предмет исследования, формулируется научная новизна.

\textbf{В первой главе} анализируется типичный применяемый подход к построению систем управления
беспилотными автомобиля и применяемые методы планирования движения. Анализируя источники по данной тематике,
можно сделать вывод, что, несмотря на большое количество архитектур систем управления беспилотными
автомобилями, можно выделить общие архитектурные концепции (рисунок \ref{img:general_arch}).

В общем виде, систему управления беспилотным автомобилем
можно разделить на следующие подсистемы:
\begin{itemize}
    \item интерфейс сенсоров, позволяющий получать данные от сенсоров;
    \item подсистема восприятия (perception), осуществляющая построение комплексной информации об
    окружающем пространстве, на основе данных от сенсоров;
    \item подсистема управления движением,осуществляющая принятие решений и построение безопасной
    и достижимой траектории и осуществляющая движение по траектории, формирования управляющих сигналов,
    таких как угол поворота руля, газ, тормоз. 
\end{itemize}

\begin{figure}[h]
    \centering
    \includegraphics[width=\linewidth]{images/general_arch}
    \caption{Обобщенная абстракция архитектуры управления автономными автомобилями}
    \label{img:general_arch}
\end{figure}

Процесс планирования движения и принятия решений в современных беспилотных автомобилях обычно представлен в виде
иерархии: планирование маршрута (route planning), принятие решений (behaviour planning, decision making), локально планирование
(local motion planning) и управление с обратной связью.

На верхнем уровне осуществляется планирование маршрута по дорожной сети. Затем следуют уровень планирования
поведения, который принимает решения и формирует локальные навигационные задачи, которые приближают автомобиль
к выполнению высокоуровневой задачи и удовлетворяют правилам дорожного движения. Затем локальный планировщик
формирует непрерывный кинематически и динамически достижимый путь в обход препятствий в окружающем пространстве, который выполняет локальную навигационную задачу. Система
управления с обратной связью осуществляет выполнение запланированного движения и коррекцию ошибок. 

В данной работе основной акцент делается на планировании локального движения. Рассматривается ряд распространенных
методов планирования движения: метод планирования на графах, в частности, метод клеточной декомпозиции, случайные 
(sample-based) методы, основанные на алгоритме Rapidly Exploring Random Trees (RRT) и его модификациях, методы 
интерполяции траектории с помощью кривых.

\textbf{Во второй главе} рассматривается проектирование системы управления движением беспилотного автомобиля. 

Для планирования локальной траектории траектории выбран метод интерполяции кривыми, а именно, полиномами
пятого порядка. Метод интерполяции выбран по причине его детерминированности и возможности находить оптимальное
или близкое к нему решение. 

Планирование траектории осуществляется в подвижной системе координат (с.к. Френе \hl{TODO: перевод}), 
движущейся по идеальной траектории, как показано на рисунке \ref{img:junior_frenet_frame}. В качестве
идеальной траектории  выступает центр полосы движения. Это является распространенным подходом к
планированию движения и позволяет рассмотреть независимо планирование поперечного движения и
продольного движения (профиля скорости).

\begin{figure}[h]
    \centering
    \includegraphics[width=0.7\textwidth]{images/junior_frenet_frame}
    \caption{Система координат для планирование траектории}
    \label{img:junior_frenet_frame}
\end{figure}

Для нахождения оптимальной траектории, определена функция стоимости, основанная на минимизации рывков
(jerk), т.е. производной ускорения. Такая функция стоимости позволит планировать траектории, избегающие
излишнего количества маневров или резких маневров. Функции стоимости определяются независимо для
поперечного (\ref{eq:cost_lat}) и продольного (\ref{eq:cost_lon}) движения.

\begin{equation}
    \label{eq:cost_lat}
    J_d = k_{jd}\int_0^S{\dddot{d}(s)ds} + k_{sd}S + k_{dd}\big(s(T)\big)^2
\end{equation}
\begin{equation}
    \label{eq:cost_lon}
    J_s = k_{js}\int_0^T{\dddot{s}(t)dt} + k_{ts}T + k_{ss}\big[s_{target} - s(T)\big]^2
\end{equation}

\noindent\begin{tabularx}{\linewidth}{lllX}
    где & $s(t)$         &~---& продольная траектория, \\
        & $d(t)$         &~---& поперечная траектория, \\
        & $T$            &~---& время маневра, \\
        & $s_{target}$   &~---& целевое продольное положение, \\
        & $k_{jd}, k_{sd}, k_{dd}$ &~---& весовые коэффициенты для поперечного движения,\\
        & $k_{js}, k_{ts}, k_{ss}$ &~---& весовые коэффициенты для продольного движения.
\end{tabularx}

При выборе оптимальной траектории необходимо учитывать ограничения на максимальную скорость, ускорение,
угол поворота, накладываемые кинематикой и динамикой автомобиля, а также выбирать траекторию, не 
пересекающуюся с препятствиями. Оптимизация с учетом этих ограничений является затруднительной и 
типичным подходом является формирование набора траекторий и выбора оптимальной. Происходит формирование
набора траекторий путем варьирования конечных условий, а затем выбор наиболее оптимальной из них, 
удовлетворяющей ограничениям. Пример формирования набора траекторий и выбора оптимальной приведен на
рисунке \ref{img:trajectory_planning_example}.

\begin{figure}[h]
    \centering
    \includegraphics[width=0.7\textwidth]{images/trajectory_planning_example}
    \caption{Пример формирования траекторий и выбора оптимальной}
    \label{img:trajectory_planning_example}
\end{figure}

Для реализации рулевого управления с обратной связью применена упрощенная модель на базе
модель МакАдама.

Управляющий сигнал для рулевого управления получается с помощью пропорционального (П) регулятора 
из угла между текущим направлением движения автомобиля (вектором скорости) и направлением на 
точку траектории на некотором расстоянии впереди. Это позволяет учитывать будущие изменения 
в траектории и реагировать на них заранее. 

По причине того, что траектория представлена в виде большого количества плотно расположенных
точек, выбор необходимой точки осуществляется следующим образом: в начале находится ближайшая к автомобилю
точка на кривой, а затем от этой точки вперед по траектории перебираются точки, до тех пор, пока
расстояние от очередной точки до положения автомобиля не будет больше заданного (рисунок \ref{img:path_moving}). 

\begin{figure}[h]
    \centering
    \includegraphics[width=0.5\textwidth]{images/path_moving}
    \caption{Пример формирования траекторий и выбора оптимальной}
    \label{img:path_moving}
\end{figure}


\referchapter{ПЕРСПЕКТИВНЫЕ НАПРАВЛЕНИЯ РАЗВИТИЯ РАБОТЫ}

\referchapter{ПУБЛИКАЦИИ ПО ТЕМЕ ДИССЕРТАЦИИ}


\end{document}